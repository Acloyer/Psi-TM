% TikZ diagram snippet for the separation stack
% Requires in main preamble: \usepackage{tikz} and \usetikzlibrary{positioning}
\begin{tikzpicture}[node distance=4.0cm]
  % Nodes: models
  \node[draw, rounded corners, fill=blue!5, minimum width=3.5cm, minimum height=1.2cm, align=center] (M) {\textbf{$\Psi$-Machine}};
  \node[draw, rounded corners, fill=green!5, minimum width=3.5cm, minimum height=1.2cm, align=center, right=of M] (T) {\textbf{$\Psi$-Decision Tree}};
  \node[draw, rounded corners, fill=orange!10, minimum width=3.5cm, minimum height=1.2cm, align=center, right=of T] (C) {\textbf{IC-Circuit}};

  % Base bridges (bidirectional) with explicit losses
  \draw[<->] (M) -- node[above]{\small $(d^{a})(\log_{2} n)^{b}$} (T);
  \draw[<->] (T) -- node[above]{\small $(k^{c})(\log_{2} n)^{\ell}$} (C);

  % Composed corollary (dashed)
  \draw[<->, dashed] (M) to[bend left=15] node[below]{\small $(d^{a})(k^{c})(\log_{2} n)^{b+\ell}$} (C);

  % Title/notes
  \node[above=1.2cm of T, align=center] (title) {\large \textbf{Bridges with Explicit Losses}};
  \node[below=1.3cm of T, align=center] (notes) {\small Losses are multiplicative; logs are base 2. Composition matches algebraic product.};
\end{tikzpicture}

