% Copyright (c) 2025 Rafig Huseynzade. All Rights Reserved.
% Licensed under CC BY-NC-ND 4.0
% Original work - do not copy without attribution

\documentclass[11pt]{article}
\usepackage{amsmath,amssymb,amsthm}
\usepackage{mathtools}
% \usepackage{geometry}
% \usepackage{hyperref}

\geometry{margin=1in}

\newtheorem{definition}{Definition}
\newtheorem{theorem}{Theorem}
\newtheorem{lemma}{Lemma}
\newtheorem{corollary}{Corollary}
\newtheorem{proposition}{Proposition}

\title{Formal Definition of the Psi-TM Model:\\
Minimal Introspective Computational Model}
\author{Mathematical Foundation}
\date{\today}

\begin{document}

\maketitle

\section{Introduction}

In this work, we formally define the computational model \textbf{Psi-TM} (Psi-Turing Machine) as a strict continuation of the Structurally-Aware Turing Machines (SA-TM) concept. The Psi-TM model is characterized by minimal introspection while preserving the ability to partially bypass complexity barriers.

\section{Formal Definition of Psi-TM}

\subsection{Basic Components}

\begin{definition}[Psi-TM Alphabet]
Let $\Sigma$ be a finite alphabet, $\Gamma = \Sigma \cup \{B\}$ be the extended alphabet, where $B$ is the blank symbol. The set of states $Q = Q_{std} \cup Q_{psi}$, where:
\begin{itemize}
\item $Q_{std}$ -- standard Turing machine states
\item $Q_{psi}$ -- introspective states with limited access to structure
\end{itemize}
\end{definition}

\begin{definition}[Psi-TM Configuration]
A configuration $\mathcal{C}$ of a Psi-TM is a tuple:
$$\mathcal{C} = (q, \alpha, \beta, \psi)$$
where:
\begin{itemize}
\item $q \in Q$ -- current state
\item $\alpha \in \Gamma^*$ -- tape content to the left of the head
\item $\beta \in \Gamma^*$ -- tape content to the right of the head
\item $\psi \in \Psi_k$ -- introspective state, where $\Psi_k$ is the set of introspective metadata of depth $\leq k$
\end{itemize}
\end{definition}

\begin{definition}[Introspective Function]
The function $\iota_k: \Gamma^* \times \Gamma^* \times \mathbb{N} \to \Psi_k$ defines introspective metadata:
$$\iota_k(\alpha, \beta, k) = \psi$$
where $k$ is the limitation on introspection depth, and $\psi$ contains only structural information of depth at most $k$.
\end{definition}

\subsection{Transition Function}

\begin{definition}[Psi-TM Transition Function]
The transition function $\delta: Q \times \Gamma \times \Psi_k \to Q \times \Gamma \times \{L, R, S\}$ is defined as:
$$\delta(q, a, \psi) = (q', b, d)$$
where:
\begin{itemize}
\item $q, q' \in Q$
\item $a, b \in \Gamma$
\item $d \in \{L, R, S\}$ -- head movement direction
\item $\psi \in \Psi_k$ -- current introspective metadata
\end{itemize}
\end{definition}

\subsection{Introspection Constraints}

\begin{definition}[k-Limited Introspection]
Introspection is called k-limited if for any $\alpha, \beta \in \Gamma^*$:
$$|\iota_k(\alpha, \beta, k)| \leq f(k)$$
where $f: \mathbb{N} \to \mathbb{N}$ is a polynomial function, and $\iota_k$ cannot access the complete tape content, only structural patterns of depth $k$.
\end{definition}

\section{Basic Properties of Psi-TM}

\subsection{Equivalence to Standard Turing Machines}

\begin{theorem}[Computational Equivalence]
For any standard Turing machine $M$, there exists an equivalent Psi-TM $M_{psi}$ with k-limited introspection, where $k = O(1)$.
\end{theorem}

\begin{proof}
Let $M = (Q, \Sigma, \Gamma, \delta, q_0, q_{accept}, q_{reject})$ be a standard Turing machine.

We construct $M_{psi} = (Q_{psi}, \Sigma, \Gamma, \delta_{psi}, q_0, q_{accept}, q_{reject}, \iota_k)$ as follows:

1. $Q_{psi} = Q \cup Q_{psi}$, where $Q_{psi} = \emptyset$ initially
2. $\iota_k(\alpha, \beta, k) = \emptyset$ for all $\alpha, \beta, k$ (empty introspection)
3. $\delta_{psi}(q, a, \emptyset) = \delta(q, a)$ for all $q \in Q_{std}$ where $\delta_{psi}: Q \times \Gamma \times \Psi_k \to Q \times \Gamma \times \{L, R, S\}$ and $\delta: Q \times \Gamma \to Q \times \Gamma \times \{L, R, S\}$ is the standard TM transition function

\textbf{Simulation Verification:} $M_{psi}$ simulates $M$ step-by-step because introspection is not used in standard states, and the transition function $\delta_{psi}$ reduces to $\delta$ when $\psi = \emptyset$.

Conversely, any Psi-TM can be simulated by a standard Turing machine by explicitly encoding introspective metadata in the state.
\end{proof}

\subsection{Bypassing Complexity Barriers}

\begin{theorem}[Partial Barrier Bypass]
There exist problems $L$ such that:
\begin{enumerate}
\item $L \in \text{PSPACE}$ for standard Turing machines
\item $L \in \text{Psi-P}_k$ for Psi-TM with suitable introspection
\end{enumerate}
\end{theorem}

\begin{proof}
Consider the Structural Pattern Recognition (SPR) problem:

\textbf{Definition of SPR:} Given a string $w \in \Sigma^*$, determine if it contains a pattern $P$ with nesting depth at most $k$.

For standard Turing machines, this problem requires $\Omega(n^k)$ time, as it is necessary to track $k$ levels of nesting.

For Psi-TM with k-limited introspection:
1. $\iota_k(w, \varepsilon, k)$ provides structural information about patterns of depth $\leq k$
2. The machine can make decisions based on introspective metadata
3. Execution time: $O(n)$

Thus, SPR $\in \text{Psi-P}_k$ for Psi-TM, but $\notin \text{P}$ for standard Turing machines (under standard complexity assumptions).
\end{proof}

\subsection{Minimality of Introspection}

\begin{theorem}[Minimality of Introspection]
If Psi-TM introspection is limited to a constant $k = O(1)$, then the model preserves equivalence to standard Turing machines in computational power.
\end{theorem}

\begin{proof}
Let $M_{psi}$ be a Psi-TM with k-limited introspection, where $k = O(1)$.

We show that $M_{psi}$ can be simulated by a standard Turing machine $M$ with polynomial slowdown:

1. State of $M$ encodes: $(q, \alpha, \beta, \psi)$
2. Size of $\psi$ is bounded by $f(k) = O(1)$
3. Each step of $M_{psi}$ is simulated in $O(1)$ steps of $M$
4. Total simulation time: $O(T(n))$, where $T(n)$ is the running time of $M_{psi}$

Conversely, any standard Turing machine can be simulated by a Psi-TM with empty introspection without slowdown.
\end{proof}

\section{Connection to SA-TM}

\begin{proposition}[Connection to SA-TM]
Psi-TM is a strict subset of SA-TM, where introspection is limited to constant depth.
\end{proposition}

\begin{proof}
Let $M_{sa}$ be an SA-TM with full introspection.

We construct an equivalent Psi-TM $M_{psi}$:
1. $\iota_k(\alpha, \beta, k) = \iota_{sa}(\alpha, \beta) \cap \Psi_k$, where $\Psi_k$ is metadata of depth $\leq k$
2. $\delta_{psi}(q, a, \psi) = \delta_{sa}(q, a, \psi)$ for $\psi \in \Psi_k$ where $\delta_{psi}: Q \times \Gamma \times \Psi_k \to Q \times \Gamma \times \{L, R, S\}$ and $\delta_{sa}$ is the SA-TM transition function

The constraint $k = O(1)$ ensures minimality of introspection while preserving the ability to bypass complexity barriers.
\end{proof}

\section{Complexity Classes}

\begin{definition}[Psi-P Class]
The class $\text{Psi-P}_k$ consists of languages recognizable by Psi-TM with k-limited introspection in polynomial time.
\end{definition}

\begin{definition}[Psi-NP Class]
The class $\text{Psi-NP}_k$ consists of languages with polynomial-time verifiable certificates using Psi-TM with k-limited introspection.
\end{definition}

\begin{definition}[Psi-PSPACE Class]
The class $\text{Psi-PSPACE}_k$ consists of languages recognizable by Psi-TM with k-limited introspection using polynomial space.
\end{definition}

\begin{theorem}[Class Hierarchy]
For any $k_1 < k_2 = O(1)$:
$$\text{Psi-P}_{k_1} \subseteq \text{Psi-P}_{k_2} \subseteq \text{PSPACE}$$
$$\text{Psi-NP}_{k_1} \subseteq \text{Psi-NP}_{k_2} \subseteq \text{NPSPACE}$$
$$\text{Psi-PSPACE}_{k_1} \subseteq \text{Psi-PSPACE}_{k_2} \subseteq \text{EXPSPACE}$$
\end{theorem}

\section{Conclusion}

The Psi-TM model represents a rigorous mathematical foundation for a minimal introspective computational model that:

1. Preserves equivalence to standard Turing machines
2. Provides partial bypass of complexity barriers
3. Minimizes introspection to constant depth
4. Formally establishes connection to SA-TM

This model opens new directions in computational complexity theory and formal automata theory.

\end{document} 