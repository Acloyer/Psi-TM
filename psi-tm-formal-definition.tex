% File content starts here - NO preamble

% Local copy of the one-step information budget lemma for standalone compilation
\begin{lemma}[One-Step Information Budget and Counting]
  \label{lem:one-step-budget}
  Assumes the restricted regime (deterministic, single pass, no advice, no randomness) and uses Table~\ref{tab:iota-spec}.
For any input $x\in\{0,1\}^n$ and any Psi-TM of depth $d$, a single call to $\iota_d$ returns an output $y\in\{0,1\}^{\le \B(d,n)}$ and hence at most $\B(d,n)$ fresh bits of introspective information. In particular, the set of possible outcomes of one call has cardinality at most $2^{\B(d,n)}$. Moreover, across $t$ computation steps and any adaptive strategy, the cumulative fresh introspective information is at most $t \cdot \B(d,n)$, and the number of possible length-$t$ outcome sequences is at most $2^{t \cdot \B(d,n)}$.
  \end{lemma}
  
  % Local copy of selector indistinguishability lemma for standalone compilation
  \begin{lemma}[Selector indistinguishability at depth d]
  \label{lem:selector-indistinguishability}
  Assumes the restricted regime (deterministic, single pass, no advice, no randomness) and uses Table~\ref{tab:iota-spec}.
  For any $d$, there exist inputs $x,x'$ with structural depths $d$ and $d+1$ such that for the same configuration $\mathcal{C}$ on inputs of length $n$, every selector over $\mathrm{decode}_d(\iota_d(\mathcal{C},n))$ returns identical outputs on $x$ and $x'$ within one step.
  \end{lemma}
  
  \section{Introduction}\label{sec:introduction}
  
  In this work, we formally define the computational model \textbf{Psi-TM} (Psi-Turing Machine) as a continuation of the Structurally-Aware Turing Machines (SA-TM) concept with minimal introspection. The Psi-TM model is characterized by selectors-only introspection semantics and explicit information budgets. Barrier statements are conservative: oracle-relative where proved, partial/conditional otherwise.
  
  \section{Formal Definition of Structural Depth}
  
  \subsection{Binary Tree Representation}
  
  \begin{definition}[Binary Tree]
  A binary tree $T$ is a finite tree where each node has at most two children. We denote:
  \begin{itemize}
  \item $\text{root}(T)$ -- the root node of $T$
  \item $\text{left}(v)$ -- the left child of node $v$ (if exists)
  \item $\text{right}(v)$ -- the right child of node $v$ (if exists)
  \item $\text{leaf}(T)$ -- the set of leaf nodes in $T$
  \item $\text{depth}(v)$ -- the depth of node $v$ (distance from root)
  \item $\text{depth}(T) = \max_{v \in T} \text{depth}(v)$ -- the depth of tree $T$
  \end{itemize}
  \end{definition}
  
  \begin{definition}[Parsing Tree]
  For a string $w \in \{0,1\}^*$, a parsing tree $T_w$ is a binary tree where:
  \begin{itemize}
  \item Each leaf is labeled with a symbol from $\{0,1\}$
  \item Each internal node represents a structural composition
  \item The concatenation of leaf labels in left-to-right order equals $w$
  \end{itemize}
  \end{definition}
  
  \subsection{Formal Structural Depth Definition}
  
  \begin{definition}[Formal Structural Depth]
  For a string $w \in \{0,1\}^*$, the structural depth $d(w)$ is defined as:
  $$d(w) = \min_{T_w} \text{depth}(T_w)$$
  where the minimum is taken over all possible parsing trees $T_w$ for $w$.
  
  \textbf{Base cases:}
  \begin{itemize}
  \item $d(\varepsilon) = 0$ (empty string)
  \item $d(0) = d(1) = 0$ (single symbols)
  \end{itemize}
  
  \textbf{Recursive case:}
  For $|w| > 1$, $d(w) = \min_{w=uv} \{1 + \max(d(u), d(v))\}$ where the minimum is taken over all binary partitions of $w$.
  \end{definition}
  
  \begin{lemma}[Well-Definedness of Structural Depth]
  The structural depth function $d: \{0,1\}^* \to \mathbb{N}$ is well-defined and computable.
  \end{lemma}
  
  \begin{proof}
  \textbf{Well-Definedness:}
  \begin{enumerate}
  \item For strings of length $\leq 1$, $d(w)$ is explicitly defined
  \item For longer strings, the minimum exists because:
    \begin{itemize}
    \item The set of possible partitions is finite (at most $n-1$ partitions for length $n$)
    \item Each partition yields a finite depth value
    \item The minimum of a finite set of natural numbers exists
    \end{itemize}
  \end{enumerate}
  
  \textbf{Computability:}
  We provide a dynamic programming algorithm. The algorithm is presented below.
  
  \textbf{Correctness:}
  \begin{enumerate}
  \item Base cases are handled correctly
  \item For each substring $w[i:j]$, we try all possible binary partitions
  \item The algorithm computes the minimum depth over all parsing trees
  \item Time complexity: $O(n^3)$ due to three nested loops
  \end{enumerate}
  \end{proof}
  
  \begin{figure}[ht]
  \framebox[\textwidth]{\begin{minipage}{0.95\textwidth}
  \textbf{Algorithm:} Structural Depth Computation
  \begin{enumerate}
  \item \textbf{Input:} String $w = w_1w_2\ldots w_n$
  \item \textbf{Output:} Structural depth $d(w)$
  \item Initialize $dp[i][j] = 0$ for all $i \leq j$
  \item \textbf{for} $i = 1$ to $n$ \textbf{do}
    \begin{enumerate}
    \item $dp[i][i] = 0$ \quad // Base case: single symbols
    \end{enumerate}
  \item \textbf{for} $\text{len} = 2$ to $n$ \textbf{do}
    \begin{enumerate}
    \item \textbf{for} $i = 1$ to $n-\text{len}+1$ \textbf{do}
      \begin{enumerate}
      \item $j = i + \text{len} - 1$
      \item $dp[i][j] = \infty$
      \item \textbf{for} $k = i$ to $j-1$ \textbf{do}
        \begin{enumerate}
        \item $dp[i][j] = \min(dp[i][j], 1 + \max(dp[i][k], dp[k+1][j]))$
        \end{enumerate}
      \end{enumerate}
    \end{enumerate}
  \item \textbf{return} $dp[1][n]$
  \end{enumerate}
  \end{minipage}}
  \end{figure}
  
  \section{Formal Definition of Psi-TM}
  
  \subsection{Basic Components}
  
  \begin{definition}[Psi-TM Alphabet]
  Let $\Sigma$ be a finite alphabet, $\Gamma = \Sigma \cup \{B\}$ be the extended alphabet, where $B$ is the blank symbol. The set of states $Q = Q_{std} \cup Q_{psi}$, where:
  \begin{itemize}
  \item $Q_{std}$ -- standard Turing machine states
  \item $Q_{psi}$ -- introspective states with limited access to structure
  \end{itemize}
  \end{definition}
  
  \begin{definition}[Psi-TM Configuration]
  A configuration $\mathcal{C}$ of a Psi-TM is a tuple:
  $$\mathcal{C} = (q, \alpha, \beta, \psi)$$
  where:
  \begin{itemize}
  \item $q \in Q$ -- current state
  \item $\alpha \in \Gamma^*$ -- tape content to the left of the head
  \item $\beta \in \Gamma^*$ -- tape content to the right of the head
  \item $\psi \in \Psi_d$ -- introspective state, where $\Psi_d$ is the set of introspective metadata of depth $\leq d$
  \end{itemize}
  \end{definition}
  
  \subsection{Formal Introspection Functions}
  
  \paragraph{Selectors as views over $\iota_d$.}
  All introspective access is via $y=\iota_d(\mathcal{C},n)$ and selectors $\mathrm{VIEW\_STATE}(y)$, $\mathrm{VIEW\_HEAD}(y)$, and $\mathrm{VIEW\_WIN}(y,d')$ applied to $\mathrm{decode}_d(y)$. Any legacy $\texttt{INT\_*}$ notation is an alias for a selector over $\mathrm{decode}_d(\iota_d(\mathcal{C},n))$.
  
  \subsection{Transition Function}
  
  \begin{definition}[Psi-TM Transition Function]
  The transition function $\delta: Q \times \Gamma \times \Psi_d \to Q \times \Gamma \times \{L, R, S\}$ is defined as:
  $$\delta(q, a, \psi) = (q', b, d)$$
  where:
  \begin{itemize}
  \item $q, q' \in Q$
  \item $a, b \in \Gamma$
  \item $d \in \{L, R, S\}$ -- head movement direction
  \item $\psi \in \Psi_d$ -- current introspective metadata
  \end{itemize}
  \end{definition}
  
  \subsection{Introspection Constraints}
  
  \begin{definition}[d-Limited Introspection]
  For a configuration $\mathcal{C}$ on an input of length $n$, a single introspection call yields the codeword $y=\iota_d(\mathcal{C},n)$. Its length is bounded by $\B(d,n)$ and $\mathrm{decode}_d(y)$ exposes only depth-$\le d$ tags (Lemma~\ref{lem:one-step-budget}).
  \end{definition}
  
\subsection{Introspection Interface \texorpdfstring{$\iota_j$}{iota-j} (Model Freeze)}
\label{sec:model-freeze}
  
  \begin{table}[t]
  \centering
\caption{Interface specification for $\iota_j$ (single source of truth). Complete definition of inputs, outputs, per-step bit budget $B(d,n) = c \cdot d \cdot \log n$, call placement constraints, and transcript accounting. All lemmas and theorems reference this specification rather than duplicating it.}
  \label{tab:iota-spec}
  \small  % уменьшаем шрифт
  \begin{tabular}{@{}p{3.5cm}p{10cm}@{}}  % фиксированная ширина колонок
  \toprule
  Field & Specification \\
  \midrule
  Depth index & $j \in \{1,\ldots,d\}$ \\
  Per-step bit budget & $B(d,n) = c \cdot d \cdot \log n$ with fixed $c \ge 1$ \\
  Call policy & Exactly once per computation step; payload injected into $\delta$ that step \\
  Inputs & Current state and allowed local view; no advice; no randomness \\
  Output payload & Bitstring $y \in \{0,1\}^{\le B(d,n)}$ \\
  Transcript accounting & Over $T$ steps: at most $T \cdot B(d,n)$ bits exposed via $\iota$; at most $2^{T \cdot B(d,n)}$ distinct transcripts \\
  \bottomrule
  \end{tabular}
\end{table}
  
  \begin{remark}[Depth notation]\label{rem:depth-notation}
We use $d$ as the global depth bound. Interface indexes are $j\in\{1,\dots,d\}$, and the per-step budget is $\B(d,n)=c\cdot d\cdot \log n$ with fixed $c\ge 1$.
  \end{remark}
  
  \begin{definition}[Iota injection into the transition]\label{def:iota-injection}
  Let $\mathcal{C}_t$ be the configuration at step $t$. Exactly once per step, the machine obtains a payload $y_t = \iota_j(\mathcal{C}_t, n) \in \{0,1\}^{\le \B(d,n)}$ and passes it as an auxiliary argument to the transition:
  \[
  (q_{t+1}, s_{t+1}) \;=\; \delta\bigl(q_t, s_t, x_t;\, y_t\bigr).
  \]
  Transcript accounting therefore sums to at most $T \cdot \B(d,n)$ bits over $T$ steps.
  \end{definition}
  
  % (i) Table~\ref{tab:iota-spec} is the single source of truth for \iota;
  % (ii) all \iota-using results state the restricted regime and cite the table;
  % (iii) Model Freeze (\S\ref{sec:model-freeze}) precedes all references; no 'Section ??' remains.
  
  \paragraph{Restricted Regime.}
  Unless stated otherwise, all results in v0.8 assume: deterministic; single pass over input; no advice; no randomness.
  We refer to Table~\ref{tab:iota-spec} whenever $\iota_j$ is used and write $\B(d,n)$ as specified above.
  
  \section{Information-Theoretic Limitations}
  
  \begin{lemma}[Selector indistinguishability at depth $d$]
  Assumes the restricted regime (deterministic, single pass, no advice, no randomness) and uses Table~\ref{tab:iota-spec}.
  For any $d$, there exist inputs $x,x'$ with structural depths $d$ and $d{+}1$ such that for the same configuration $\mathcal{C}$ on inputs of length $n$, every selector over $\mathrm{decode}_d(\iota_d(\mathcal{C},n))$ returns identical outputs on $x$ and $x'$ within one step.
  \end{lemma}
  
  \begin{proof}
We provide a self-contained derivation using the Budget Lemma (transcript bound $2^{T \cdot \B(d,n)}$) under the restricted regime.
  
  \textbf{Key insight:} For inputs $x,x'$ with structural depths $d$ and $d+1$, the depth-$d$ projection is identical, so selectors over $\mathrm{decode}_d(\iota_d(\mathcal{C},n))$ return identical values.
  
\textbf{Formal derivation:} By the Budget Lemma, each $\iota_d$ call has at most $2^{\B(d,n)}$ outcomes. Over $T$ steps, there are at most $2^{T \cdot \B(d,n)}$ outcome sequences. Since $T \cdot \B(d,n) \ge \log M$ where $M$ is the number of distinct selector outputs, the machine cannot distinguish between inputs that agree on depth-$d$ features.
  
  Therefore, for any $d$, there exist inputs with depths $d$ and $d+1$ that are indistinguishable by depth-$d$ introspection.
  \end{proof}
  
  \section{Basic Properties of Psi-TM}
  
  \subsection{Equivalence to Standard Turing Machines}
  
  \begin{theorem}[Computational Equivalence]
  Assumes the restricted regime (deterministic, single pass, no advice, no randomness) and uses Table~\ref{tab:iota-spec}.
  For any standard Turing machine $M$, there exists an equivalent Psi-TM $M_{psi}$ with d-limited introspection, where $d = O(1)$.
  \end{theorem}
  
  \begin{proof}
  Let $M = (Q, \Sigma, \Gamma, \delta, q_0, q_{accept}, q_{reject})$ be a standard Turing machine.
  
  We construct $M_{psi} = (Q_{psi}, \Sigma, \Gamma, \delta_{psi}, q_0, q_{accept}, q_{reject}, \iota_d)$ as follows:
  
  \begin{enumerate}
  \item $Q_{psi} = Q \cup Q_{psi}$, where $Q_{psi} = \emptyset$ initially
  \item No introspection is used (no calls to $\iota_d$)
  \item $\delta_{psi}(q, a, \emptyset) = \delta(q, a)$ for all $q \in Q_{std}$
  \end{enumerate}
  
  \textbf{Simulation Verification:} 
  $M_{psi}$ simulates $M$ step-by-step because introspection is not used in standard states, and the transition function $\delta_{psi}$ reduces to $\delta$ when $\psi = \emptyset$.
  
  \textbf{Reverse Simulation:}
  Any Psi-TM can be simulated by a standard Turing machine by explicitly encoding introspective metadata in the state. The size of $\psi$ is bounded by $f(d) \cdot n = O(n)$ for constant $d$, so the simulation requires polynomial overhead.
  \end{proof}
  
  \subsection{Barrier status (conservative)}
  
  \begin{theorem}[Conservative barrier statements]
  Assumes the restricted regime (deterministic, single pass, no advice, no randomness) and uses Table~\ref{tab:iota-spec}.
  There exist oracle-relative separations and partial/conditional results consistent with the barrier status in the barrier analysis section:
  \begin{enumerate}
\item Oracle-relative: $P^{O_\Psi}_\Psi \neq NP^{O_\Psi}_\Psi$ for a suitable oracle $O_\Psi$ (Theorem~\ref{thm:diagonal-1})
  \item Partial/conditional: statements for natural proofs and proof complexity; algebraization open/conservative
  \end{enumerate}
  \end{theorem}
  
  \begin{proof}
  Consider the Structural Pattern Recognition (SPR) problem:
  
  \textbf{Definition of SPR:} 
  Given a string $w \in \{0,1\}^*$, determine if $d(w) \leq d$.
  
  \textbf{Standard TM Complexity:}
  For standard Turing machines, this requires $\Omega(n^d)$ time, as it is necessary to track $d$ levels of nesting by explicit computation.
  
  \textbf{Psi-TM Solution (selectors-only):}
  For Psi-TM with d-limited introspection, obtain $y=\iota_d(\mathcal{C},n)$ and use selectors over $\mathrm{decode}_d(y)$ to read bounded-depth summaries; all accesses obey the budget in Lemma~\ref{lem:one-step-budget}.
  
  \textbf{Time Analysis:}
  \begin{enumerate}
  \item $\texttt{INT\_STRUCT(d)}(w)$ computation: $O(n^3)$ by the dynamic programming algorithm
  \item Pattern checking: $O(n)$ since $d = O(1)$
  \item Total time: $O(n^3)$
  \end{enumerate}
  
  Thus, SPR $\in \text{Psi-P}_d$ for Psi-TM, but requires $\Omega(n^d)$ time for standard Turing machines (under standard complexity assumptions).
  \end{proof}
  
  % Local copy of the diagonal separation theorem for standalone compilation
  \begin{theorem}[Diagonal Separation for Psi-TM]
  \label{thm:diagonal-1}
  Assumes the restricted regime (deterministic, single pass, no advice, no randomness) and uses Table~\ref{tab:iota-spec}.
  There exists an oracle $O_\Psi$ such that $P^{O_\Psi}_\Psi \neq NP^{O_\Psi}_\Psi$.
  \end{theorem}
  
  \subsection{Minimality of Introspection}
  
  \begin{theorem}[Minimality of Introspection]
  Assumes the restricted regime (deterministic, single pass, no advice, no randomness) and uses Table~\ref{tab:iota-spec}.
  If Psi-TM introspection is limited to a constant $d = O(1)$, then the model preserves equivalence to standard Turing machines in computational power.
  \end{theorem}
  
  \begin{proof}
  Let $M_{psi}$ be a Psi-TM with d-limited introspection, where $d = O(1)$.
  
  We show that $M_{psi}$ can be simulated by a standard Turing machine $M$ with polynomial slowdown:
  
  \begin{enumerate}
  \item State of $M$ encodes: $(q, \alpha, \beta, \psi)$
  \item Size of $\psi$ is bounded by $f(d) \cdot n = O(n)$ for constant $d$
  \item Each introspection call $y=\iota_d(\mathcal{C},n)$ is computed explicitly in $O(n^3)$ time
  \item Each step of $M_{psi}$ is simulated in $O(n^3)$ steps of $M$
  \item Total simulation time: $O(T(n) \cdot n^3)$, where $T(n)$ is the running time of $M_{psi}$
  \end{enumerate}
  
  \textbf{Reverse Simulation:}
  Any standard Turing machine can be simulated by a Psi-TM with empty introspection without slowdown.
  
  This establishes polynomial-time equivalence between Psi-TM with constant introspection depth and standard Turing machines.
  \end{proof}
  
  \section{Complexity Classes}
  
  \begin{definition}[Psi-P Class]
  The class $\text{Psi-P}_d$ consists of languages recognizable by Psi-TM with d-limited introspection in polynomial time.
  \end{definition}
  
  \begin{definition}[Psi-NP Class]
  The class $\text{Psi-NP}_d$ consists of languages with polynomial-time verifiable certificates using Psi-TM with d-limited introspection.
  \end{definition}
  
  \begin{definition}[Psi-PSPACE Class]
  The class $\text{Psi-PSPACE}_d$ consists of languages recognizable by Psi-TM with d-limited introspection using polynomial space.
  \end{definition}
  
  \begin{theorem}[Class Hierarchy]
  For any $d_1 < d_2 = O(1)$:
  $$\text{Psi-P}_{d_1} \subseteq \text{Psi-P}_{d_2} \subseteq \text{PSPACE}$$
  $$\text{Psi-NP}_{d_1} \subseteq \text{Psi-NP}_{d_2} \subseteq \text{NPSPACE}$$
  $$\text{Psi-PSPACE}_{d_1} \subseteq \text{Psi-PSPACE}_{d_2} \subseteq \text{EXPSPACE}$$
  \end{theorem}
  
  \begin{proof}
  \textbf{Inclusion Proof:}
  Let $L \in \text{Psi-P}_{d_1}$. Then there exists a Psi-TM $M$ with $d_1$-limited introspection that recognizes $L$ in polynomial time.
  
  We construct a Psi-TM $M'$ with $d_2$-limited introspection:
  \begin{enumerate}
  \item $M'$ simulates $M$ step-by-step
  \item For each introspection call of $M$, $M'$ performs the same introspection
  \item Since $d_1 < d_2$, all introspection calls of $M$ are valid for $M'$
  \item Time complexity remains polynomial
  \end{enumerate}
  
  \textbf{PSPACE Inclusion:}
  Any Psi-TM with constant introspection depth can be simulated by a standard Turing machine with polynomial space overhead, as shown in the minimality theorem.
  
  The same arguments apply to NP and PSPACE classes.
  \end{proof}
  
\section{Outlook — Model Freeze}
  
  The Psi-TM model represents a rigorous mathematical foundation for a minimal introspective computational model that:
  
  \begin{enumerate}
  \item Preserves equivalence to standard Turing machines
  \item Provides partial bypass of complexity barriers
  \item Minimizes introspection to constant depth
  \item Formally establishes structural depth as a computable property
  \item Provides explicit constructions for information-theoretic limitations
  \end{enumerate}
  
  This model opens new directions in computational complexity theory and formal automata theory.
  
  % End of included content